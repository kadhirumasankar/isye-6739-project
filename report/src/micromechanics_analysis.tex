\section{Micromechanics Analysis}

To derive Young's modulus in direction 1, $E_1$, we start with the total force equation, where $f$ and $m$ stand for the Fiber and Matrix respectively.

\begin{equation}
P=\sigma_f A_f + \sigma_m A_m
\end{equation}

Then we state the average stress in direction 1.

\begin{equation}
\sigma_1 = \frac{P}{A_f + A_m} = \frac{\sigma_f A_f + \sigma_m A_m}{A_f + A_m}
\end{equation}

This can be simplified to give the following:

\begin{equation}
\sigma_1 = \sigma_f \frac{1}{1+A_m/A_f} + \sigma_m \frac{A_m/A_f}{1+A_m/A_f}
\end{equation}

We can substitute the volume fraction as follows:

\begin{equation}
\frac{1}{1+A_m/A_f} = V_f = 1 - V_m
\end{equation}

Then the stress in direction 1 will be:

\begin{equation}
\sigma_1 = \sigma_f V_f + \sigma_m (1-V_f)
\end{equation}

Using the relationship between stress, strain, and Young's modulus, we get Equation \ref{eq:E1}, the Young's modulus of the composite in direction 1.

\begin{equation} \label{eq:E1}
E_1 = E_f V_f + E_m (1-V_f)
\end{equation}

To derive Young's modulus in direction 2, we realize that the stresses of the fiber and matrix are the same, but their strains are different. Thus, we know that the equation for total elongation is as follows:

\begin{equation}
\delta = \varepsilon_f h_f + \varepsilon_m h_m
\end{equation}

Similar to the procedure used for total stress, we can write total strain as:

\begin{equation}
\varepsilon_2 = \frac{\varepsilon_f h_f + \varepsilon_m h_m}{h_f + h_m} = \varepsilon_f \frac{1}{1+h_m/h_f} + \varepsilon_m \frac{h_m/h_f}{1+h_m/h_f}
\end{equation}

We can use the volume fraction equation to get the following equation for total strain:

\begin{equation}
\varepsilon_2 = \varepsilon_f V_f + \varepsilon_m (1-V_f)
\end{equation}

By plugging that into Hooke's equation, we get Equation \ref{eq:E2}, Young's modulus in direction 2.

\begin{equation} \label{eq:E2}
E_2 = (\frac{V_f}{E_f} + \frac{1-V_f}{E_m})^{-1}
\end{equation}

To find the Poisson's ratio in direction 1, we leverage the total transverse strain:

\begin{equation}
\varepsilon_2 = \varepsilon_f' V_f + \varepsilon_m' (1-V_f)
\end{equation}

Using that, we can get Equation \ref{eq:nu2}, Poisson's ratio in direction 2.

\begin{equation} \label{eq:nu2}
\nu_{21} = -\frac{\varepsilon_2}{\varepsilon_1} = \nu_f V_f + \nu_m (1-V_f)
\end{equation}

To derive the shear modulus of the composite, we use the equation for the composite shear strain.

\begin{equation}
\gamma_{12} = \gamma_f V_f + \gamma_m (1-V_f)
\end{equation}

We divide through that by the shear strain to get Equation \ref{eq:G}, the shear modulus of the material.

\begin{equation} \label{eq:G}
G_{12} = \frac{\gamma_{12}}{\sigma_{12}} = (\frac{V_f}{G_f} + \frac{1-V_f}{G_m})^{-1}
\end{equation}
