\section{Results and Discussion}

Table \ref{table:Kcrit} was created by using Equation \ref{eq:Kcrit} to find the critical stress intensity factors, $K_C$, for different values of initial crack lengths, $a_0$, and different specimen thicknesses. Using the same data points, a plot of the fracture toughness for different initial crack lengths was created, which can be seen in Figure \ref{fig:KcVSa0}. A plot of the average critical stress intensity factor against the specimen thicknesses can be seen in Figure \ref{fig:AvgKcVSThickness}, and the negative trend seen in this figure is intuitive, as average toughness decreases with increasing specimen thickness.

\begin{table}[H]
\centering
\begin{tabular}{|c|c|c|}
\hline
\textbf{Thickness}        & \textbf{$a_0$ (in)} & \textbf{$K_C$} \\ \hline
\multirow{6}{*}{0.5 in}   & 0.89           & 37965.8244     \\ \cline{2-3} 
                          & 0.986          & 37908.5895     \\ \cline{2-3} 
                          & 0.905          & 30781.3273     \\ \cline{2-3} 
                          & 1.061          & 41217.1462     \\ \cline{2-3} 
                          & 1.006          & 38510.6926     \\ \cline{2-3} 
                          & 0.851          & 34975.1817     \\ \hline
\multirow{6}{*}{0.25 in}  & 0.719          & 37955.1178     \\ \cline{2-3} 
                          & 0.9345         & 51430.7034     \\ \cline{2-3} 
                          & 0.9585         & 37061.6005     \\ \cline{2-3} 
                          & 0.82           & 48584.4641     \\ \cline{2-3} 
                          & 0.703          & 35624.8686     \\ \cline{2-3} 
                          & 0.772          & 54836.7813     \\ \hline
\multirow{6}{*}{0.125 in} & 0.987          & 57274.8526     \\ \cline{2-3} 
                          & 1.105          & 51956.9193     \\ \cline{2-3} 
                          & 1.118          & 51847.9702     \\ \cline{2-3} 
                          & 0.918          & 36215.4784     \\ \cline{2-3} 
                          & 0.819          & 46849.6807     \\ \cline{2-3} 
                          & 0.9385         & 30393.6034     \\ \hline
\end{tabular}
\caption{Table of critical stress intensity factors ($K_C$) for different initial crack lengths ($a_0$)}
\label{table:Kcrit}
\end{table}

Table \ref{table:K_IC} shows the evaluation of Equation \ref{eq:ASTM} for all of the samples. From this table, we can see that since some values of $a$ satisfy the condition for $K_{IC}$, the thicknesses of the specimens are too small to satisfy the condition. Therefore, to find $K_{IC}$, we would have to test thicker specimens.

\begin{table}[H]
\centering
\begin{tabular}{|c|c|c|}
\hline
\textbf{Thickness}        & \textbf{$a_0$ (in)} & \textbf{$2.5(\frac{K_C}{\sigma_{ys}})^2$} \\ \hline
\multirow{6}{*}{0.5 in}   & 0.89           & 1.1091     \\ \cline{2-3} 
                          & 0.986          & 1.1057     \\ \cline{2-3} 
                          & 0.905          & 0.7290     \\ \cline{2-3} 
                          & 1.061          & 1.3072     \\ \cline{2-3} 
                          & 1.006          & 1.1411     \\ \cline{2-3} 
                          & 0.851          & 0.9412     \\ \hline
\multirow{6}{*}{0.25 in}  & 0.719          & 1.1084     \\ \cline{2-3} 
                          & 0.9345         & 2.0353     \\ \cline{2-3} 
                          & 0.9585         & 1.0569     \\ \cline{2-3} 
                          & 0.82           & 1.8162     \\ \cline{2-3} 
                          & 0.703          & 0.9765     \\ \cline{2-3} 
                          & 0.772          & 2.3138     \\ \hline
\multirow{6}{*}{0.125 in} & 0.987          & 2.5241     \\ \cline{2-3} 
                          & 1.105          & 2.0771     \\ \cline{2-3} 
                          & 1.118          & 2.0684     \\ \cline{2-3} 
                          & 0.918          & 1.0092     \\ \cline{2-3} 
                          & 0.819          & 1.6888     \\ \cline{2-3} 
                          & 0.9385         & 0.7108     \\ \hline
\end{tabular}
\caption{Table showing $2.5(\frac{K_C}{\sigma_{ys}})^2$ for all of the samples}
\label{table:K_IC}
\end{table}

When examining the fracture surfaces, the fracture line is perpendicular to the loading direction, which is the plane-strain condition. Further along the specimen, the fracture line is at a 45$^\circ$ angle to the loading direction, which is the plane-stress fracture mode. This is intuitive, because 45$^\circ$ is the angle of maximum shear strain.

Plots showing the load v.s. displacement for the different specimens can be seen in Figures \ref{fig:05} to \ref{fig:0125}. It can be noted visually that the slopes of the linear parts of the curves are approximately equal.

Figure \ref{fig:compliance} shows a plot of the compliance values obtained for each specimen thickness as a function of the initial crack length. This proves the observation made earlier: that the slopes of the linear parts of the curves in Figures \ref{fig:05} to \ref{fig:0125} are approximately equal.
