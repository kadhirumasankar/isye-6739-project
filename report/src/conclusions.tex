\section{Conclusions}

For this project, the expected value and distribution of the simple ``Coins in a Pot" game was found using simulation. Code was written in Python following the rules of the game, and was executed many times. As the number of repetitions increased, the statistics began to converge to certain values, which is consistent with the Law of Large Numbers. Although the results in this report are from running the game only 1,000,000 times, with access to more computing power, many more games can be simulated, and as more games are simulated, the estimate will begin to get closer to the actual parameter.

Although the ``Coins in a Pot" game is seemingly simple, calculating its expected value by hand is a very arduous task. It is possible to find the expected value through first-step analysis, but, although this game is much simpler than a game such as Blackjack, the equation would grow rapidly and the number of terms in the equation would quickly become unmanageable. In more complicated games with more outcomes, finding the expected value would be unfeasible. In such cases, it is simpler to write code that accurately simulates a game, run the simulation millions of times, and record statistics of interest. Through this project, the importance of simulations for statistical analysis was revealed.

